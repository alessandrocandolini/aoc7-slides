\PassOptionsToPackage{table}{xcolor}
\documentclass[10pt]{beamer}
\usepackage[english]{babel}

% Beamer theme
\usetheme{metropolis}

\usepackage{smartdiagram}
\usepackage{lstautogobble}
\usepackage{listings}
\usepackage{booktabs}
\usepackage[scale=2]{ccicons}%creative commons
\setbeamercovered{transparent}%invisible by default
\usepackage{array}
\newcolumntype{L}[1]{>{\raggedright\let\newline\\arraybackslash\hspace{0pt}}m{#1}}
\newcolumntype{C}[1]{>{\centering\let\newline\\arraybackslash\hspace{0pt}}m{#1}}
\newcolumntype{R}[1]{>{\raggedleft\let\newline\\arraybackslash\hspace{0pt}}m{#1}}

\usepackage{pgfplots}
\usepgfplotslibrary{dateplot}
\usepackage{tikz}
\usetikzlibrary{positioning,chains,fit,shapes,calc}
\usetikzlibrary{positioning,chains,fit,shapes,calc,automata,positioning}
\usepackage{fancyvrb}

% To include metapost files.
\usepackage{ifpdf}                        % To check if pdflatex is used.
\ifpdf
  \DeclareGraphicsRule{*}{mps}{*}{}
\fi

% Define path for images.
\graphicspath{{./}, {./Images/}}

% few useful commands
\providecommand{\ie}{i.\,e.}
\providecommand{\Ie}{I.\,e.}
\providecommand{\eg}{e.\,g.}
\providecommand{\Eg}{E.\,g.}

% Basic listing setting (eg, autogobble to remove left margin)
\lstset{basicstyle=\ttfamily, autogobble}

% metropolis template settings
\metroset{block=fill}
\metroset{titleformat frame=smallcaps}

\title{From imperative code to recursion schemes}
%\subtitle{}

\date{\today}
\author[Alessandro Candolini]{Alessandro Candolini}
%\institute{}
% \titlegraphic{\hfill\includegraphics[height=1.5cm]{logo/logo}}

\begin{document}

\maketitle

\begin{frame}{Agenda}
  \setbeamertemplate{section in toc}[sections numbered]
  \tableofcontents[hideallsubsections]
\end{frame}

\section{Isomorphisms}
\begin{frame}[fragile]
  \begin{lstlisting}[language=scala]
  sealed class Node(
     open val name: String, 
     open val parent: Directory?
  )

  data class File(
    override val name: String, 
    override val parent: Directory?, 
    val size: Long): Node(name, parent)

  data class Directory(
    override val name: String,
    override val parent: Directory?,
    val children: MutableList<Node>
  ) : Node(name, parent)
  \end{lstlisting}
\end{frame}

\begin{frame}[fragile]
  file system is represented by an \emph{algebraic data type inductively defined}, which captures%
  \footnote{Wait next slides for cardinality questions and isomorphic and not isomorphic representations}
  \begin{itemize}
    \item the recursive nature of the filesystem 
    \item separating attributes of files and attributes of directories (\ie, size) 
  \end{itemize}
\end{frame}
\begin{frame}[fragile]
  \begin{itemize}
    \item Blend of domain modelling and implementation details 
    \item model design polluted by / coflated with elements to support the underlying implementation (\ie, leaking implementation choices to the consumer space)
  \end{itemize} 
\end{frame}

\begin{frame}[fragile]
\begin{lstlisting}[language=scala]
data class Node(
  val name: String,
  val size: Long?,
  val children: MutableList<Node>?,
  val isFile : Boolean, 
  val parent: Directory?
)
\end{lstlisting}
\end{frame}
\begin{frame}[fragile]
\begin{lstlisting}[language=scala]
sealed interface Node

data class File(
  val name: String,
  val size: Long
) : Node

data class Directory(
  val name: String,
  val children: List<Node>
) : Node
\end{lstlisting}
\end{frame}

\begin{frame}[fragile]
  \begin{lstlisting}[language=haskell]
  data FileSystem
    = File FileName Size
    | Directory DirectoryName [FileSystem]
    deriving (Eq, Show)
  \end{lstlisting}
\end{frame}
\begin{frame}[fragile]
  \begin{lstlisting}[language=scala]
  enum FileSystem:
    case File(
      name: FileName,
      size: Size)
    case Directory(
      name: DirectoryName, 
      children: List[FileSystem])
  \end{lstlisting}
\end{frame}
\begin{frame}[fragile]
  \begin{lstlisting}[language=haskell]
  data Rose a = Rose a [Rose a] 
  data Tree1 a = Leaf1 a | Node1 [Tree1 a] 
  data Tree2 a = Leaf2 a | Node2 (NonEmpty (Tree2 a)) 
  data Tree3 a = Leaf3 a | Branch (Tree3 a) (Tree3 a) 
  data Tree4 a = Empty | Node4 (a, Tree4 a , Tree4 a) 
  \end{lstlisting}
\end{frame}

\begin{frame}[fragile]
  \begin{lstlisting}[language=haskell]
  data FileSystem
    = File Size
    | Directory (Map Name FileSystem)
    deriving (Eq, Show)
  \end{lstlisting}
\end{frame}


\begin{frame}[fragile]
  \begin{itemize} 
    \item inductive types and custom data structures (``tree'' / boolean blindness)
    \item isomorphic and non-isomorphic representations 
    \item encoding of constrains (eg, uniqueness of the names)
    \item redundancy and data integrity 
  \end{itemize}
\end{frame}


\section{Algebra of operations}


\begin{frame}[fragile]
\begin{lstlisting}[language=haskell]

data FileSystem

file :: Name -> Size -> FileSystem
directory :: Name -> [FileSystem] -> FileSystem 

du :: FileSystem -> Size 
pretty :: FileSystem -> Doc 
duDir :: FileSystem -> [(Name, Size)]
\end{lstlisting}
\end{frame}


\begin{frame}[fragile]
  \begin{lstlisting}[language=haskell]
  \end{lstlisting}
\end{frame}

\begin{frame}[fragile]
  \begin{lstlisting}[language=haskell]
  \end{lstlisting}
\end{frame}
\begin{frame}[fragile]
  \begin{lstlisting}[language=haskell]
  \end{lstlisting}
\end{frame}
\begin{frame}[fragile]
  \begin{lstlisting}[language=haskell]
  \end{lstlisting}
\end{frame}
\plain{Questions?}

%\begin{frame}[allowframebreaks] {References}
% \bibliography{demo}
% \bibliographystyle{abbrv}
%\end{frame}

\end{document}
\begin{frame}[fragile]
  \begin{lstlisting}[language=haskell]
  \end{lstlisting}
\end{frame}
%% Useful for copy pasting. Would be better to just have snippets
\begin{frame}[fragile
  \frametitle{}
  \begin{lstlisting}[language=scala, basicstyle=\ttfamily]
  \end{lstlisting}
\end{frame}
