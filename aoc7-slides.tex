\PassOptionsToPackage{table}{xcolor}
\documentclass[10pt]{beamer}
\usepackage[english]{babel}

% Beamer theme
\usetheme{metropolis}

\usepackage{smartdiagram}
\usepackage{lstautogobble}
\usepackage{listings}
\usepackage{booktabs}
\usepackage[scale=2]{ccicons}%creative commons
\setbeamercovered{transparent}%invisible by default
\usepackage{array}
\newcolumntype{L}[1]{>{\raggedright\let\newline\\arraybackslash\hspace{0pt}}m{#1}}
\newcolumntype{C}[1]{>{\centering\let\newline\\arraybackslash\hspace{0pt}}m{#1}}
\newcolumntype{R}[1]{>{\raggedleft\let\newline\\arraybackslash\hspace{0pt}}m{#1}}

\usepackage{pgfplots}
\usepgfplotslibrary{dateplot}
\usepackage{tikz}
\usetikzlibrary{positioning,chains,fit,shapes,calc}
\usetikzlibrary{positioning,chains,fit,shapes,calc,automata,positioning}
\usepackage{fancyvrb}

% To include metapost files.
\usepackage{ifpdf}                        % To check if pdflatex is used.
\ifpdf
  \DeclareGraphicsRule{*}{mps}{*}{}
\fi

% Define path for images.
\graphicspath{{./}, {./Images/}}

% few useful commands
\providecommand{\ie}{i.\,e.}
\providecommand{\Ie}{I.\,e.}
\providecommand{\eg}{e.\,g.}
\providecommand{\Eg}{E.\,g.}

% Basic listing setting (eg, autogobble to remove left margin)
\lstset{basicstyle=\ttfamily, autogobble}

% metropolis template settings
\metroset{block=fill}
\metroset{titleformat frame=smallcaps}

\title{From imperative code to recursion schemes}
%\subtitle{}

\date{\today}
\author[Alessandro Candolini]{Alessandro Candolini}
%\institute{}
% \titlegraphic{\hfill\includegraphics[height=1.5cm]{logo/logo}}

\begin{document}

\maketitle

\begin{frame}{Agenda}
  \setbeamertemplate{section in toc}[sections numbered]
  \tableofcontents[hideallsubsections]
\end{frame}

\section{Warm up}
\begin{frame}[fragile]
  \frametitle{Placeholder Example}
  \begin{lstlisting}[language=scala]
    sealed trait Result[+A]
    case object Loading extends Result[Nothng]
    case class  Error(s : String) extends Result[Nothing]
    case class Success[+A](a : A) extends Result[A]

    def f[A] : Option[A] => Result[A] = {
      case Some(a) => Success(a)
      case _       => Error("invalid")
    }
  \end{lstlisting}
\end{frame}


\plain{Questions?}

%\begin{frame}[allowframebreaks] {References}
% \bibliography{demo}
% \bibliographystyle{abbrv}
%\end{frame}

\end{document}
%% Useful for copy pasting. Would be better to just have snippets
\begin{frame}[fragile]
  \frametitle{}
  \begin{lstlisting}[language=scala, basicstyle=\ttfamily]
  \end{lstlisting}
\end{frame}
